\section*{Opgave 8}

I opgave teksten betragtes et array ad længde $n-1$ 
som indeholder alle heltal mellem i intervallet $\left[0;n\right]$ 
foruden et på frhånd ukendt tal. Da ønskes an algoritme som 
kan finde lige netop det tal som mangler.


\section*{8.1}

I første opgave findes det manglende tal ved brug af en ny liste.
Denne første algoritme laver et nyt array af længde $n$. Dette nye
array's elementer er af typen 'bool' og sættes som udgangspunkt til 'false'.
Vi kalder det nye array B og input array A.\\

Algoritmen løber da A igennem, og for hver iteration $i$ sættes 
$B \left[ A \left[ i \right] \right]$ til 'true'.\\

Efter dette gennemløb kan $B$ gennemløbes og når en indgang 
har værdien 'false' returneres dette indeks som det manglende tal.\\

Algoritmen er implementeret i C++ og kan ses nedenfor.


\begin{lstlisting}[language=C++]
int find_missing(int *A, int n){
  
  /*
    c0 time for memory alloc
    n time for setting all to false
  */
  std::vector<bool> found(n, false);


  /*
    n-1 for running through the whole array
  */
  for (int i = 0; i < n-1; i++){
    found[A[i]] = true;
  }

  /*
    n for running through the whole array
  */
  for (int i = 0; i < found.size(); i++){
    if (!found[i]){
      return i;
    }
  }
  return 0;
}
\end{lstlisting}


Algoritmmen bruger følgende tid på operationer

\begin{center}
\begin{tabular}{| l | l | l |}
\hline
$c0$ & konstant tid på allokering af found array'et & linje 7\\
\hline
$n$ & $n$ tid på at sætte alle indgange til 'false' & linje 7\\
\hline
$n-1$ & $n-1$ På at sætte 'true' i alle fundne tals indgange & linje 13-14\\
\hline
$n$ & op til $n$ på at finde det tal som mangler & linje 20-22\\
\hline
\end{tabular}
\end{center}



Derved er den samlede tid 

\begin{align*}
T &= c0 + n + n-1 + n = c0 + 3n -1\\
T &= \Theta (n)
\end{align*}


Mens hukommelses forbruget samtidig er $\Theta (n)$


\section*{8.2}

En algoritme der udføre samme opgave implementeres. Denne skal 
blot bruge $\Theta (n^2)$ tid og $\Theta (1)$ hukommelse.\\

I stedet for at lave et nyt array med $n$ indgange, itereres der blot 
over alle heltal i intervalet $\left[0 ; n\right]$, hvor der for hvert
af disse iteration ses om det pågældende tal findes.\\

Algoritmen er igen implkementeret i C++ og kan ses neden på næste side.


\begin{lstlisting}[language=C++]
int find_missing_2(int *A, int n){

  /*
    only extra veriables
    bool: 1 byte
    int : 4 bytes
    total: 5 bytes - always
  */
  int expected_lowest = 0;
  bool found = false;

  for (int i = 0; i < n - 1; i++){
    found = false;
    for (int i = 0; i < n - 1; i++){
      if (A[i] == expected_lowest){
        found = true;
        expected_lowest++;
        break;
      }
    }
    if (!found){
      return expected_lowest;
    }
  }
  return 0;
}
\end{lstlisting}

\section*{8.3}

Denne gang udnyttes det at summen af alle tal mellem i intervallet
$\left[ 0; n-1 \right]$ kan skrives som $\frac{n\left( n-1 \right)}{2}$.
Algoritmen summerer da alle tal i $A$ og trækker det fra den udregnede sum.
Forskellen mellem de to tal vil da netop have samme værdi som det tal 
der magler.\\

Da input array'et kun skal løbes igennem en enkelt gang, og de øvrige operationer
tager konstant tid, vil det endelige tidforbrug være $\Theta (n)$ mens
ekstra hukommelses forbrug er  $\Theta (1)$\\

Algoritmen kan ses neden for.

\begin{lstlisting}[language=C++]
int find_missing_3(int *A, int n){
  int sum_all = (n*(n - 1)) / 2;
  int sum_real = 0;
  for (int i = 0; i < n - 1; i++){
    sum_real += A[i];
  }
  return sum_all-sum_real;
}
\end{lstlisting}




