\section{Assignment 1}

\subsection{Problem}

The Alrothm in this assignment should compute the size of a
social network. The data format is specified as
three components given in a text file. The first line of the file is
a list $P$ of all names in the network thus containing all
persons in the network as $p_0$ through $p_n$. Followed by the line of names, 
$V$ lines should occur. These lines should be composed of two integer 
values $i$ and $j$, and describes that $p_i$ and $p_j$ is in a friendship 
relation. The final cmponent tells what kind of computation the program should make.
In this case, the final line shoudl be \textit{stoerrelse}.


\subsection{Algorithm}

In the sourcecode, the function to compute this is called \textit{calc\_size},
and has the following pseudocode

\begin{figure}[ht]
\hrule
\vspace{0.2cm}
{\centering  \textit{calc\_size}}
\vspace{0.2cm}
\hrule
\begin{algorithmic}

\Function{calc\_size}{P, V} 
\State $a = \text{Count all spaces in P}$
\State $b = \text{Get length of V}$
\State $\text{Print "a b"}$ 
\EndFunction
\end{algorithmic}
\hrule
\end{figure}

The parameter $P$ is seen as the line with all names in it, and
$V$ is the list if relations. These are first read before the 
function is called. The computations are fairly simple. The
seperator in $P$ is a space character, so the amount of 
persons is equal to the amount of spaces. The number of relations
is also just found as the number of elements in $V$.

\subsection{Complexity}

finding the amount of persons in the network is based on 
scanning the first line for spaces character by character.
Thus, if there are $N$ characters in the line, it takes $N$
operations. The number of realations, that is the size of $V$
takes a constant number of operations per relation element. 
Albiet not present in the algorithm itself, this would take
$V$ operations to compute. The time complexity of the algorithm
is then.

\begin{align*}
T(P, V, N) &= \theta \left( V \right) + \theta \left( N \right)\\
           &= \theta \left( V + N \right)
\end{align*}




