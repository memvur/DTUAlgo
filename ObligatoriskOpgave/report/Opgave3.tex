\section{Assignment 3}

\subsection{Problem}

The algortihm in this assignment should find a chain of 
freinds from a given node in a social network. The
length of the chain $t, t \geq 0$ is given as an integer, and starts 
from an arbitrary node in the network $p$. The chain then 
consists of all persons that can be reached from $p$ through
at most $t$ relationship connections. An example of such a chain
can be seen in figure \ref{fig:tvenner}


\begin{figure}[ht]
\centering
\begin{tikzpicture}[scale=0.7, auto,swap]
    \coordinate (c0) at (3, 0);
    \coordinate (c1) at (-3, -3);
    \coordinate (c2) at (0, 0);
    \coordinate (c3) at (3, -3);
    \coordinate (c4) at (-3, 3);
    \coordinate (c5) at (3, 3);
    \coordinate (c6) at (0, 6);

    \node[circle, draw, minimum size=\noderad] (v0) at (c0) {0};
    \node[circle, draw, minimum size=\noderad] (v1) at (c1) {1};
    \node[circle, draw, minimum size=\noderad] (v2) at (c2) {2};
    \node[circle, draw, minimum size=\noderad] (v3) at (c3) {3};
    \node[circle, draw, minimum size=\noderad] (v4) at (c4) {4};
    \node[circle, draw, minimum size=\noderad] (v5) at (c5) {5};
    \node[circle, draw, minimum size=\noderad] (v6) at (c6) {6};

    \draw [-] (v1) -- (v2);
    \draw [-] (v1) -- (v4);

    \draw [-] (v2) -- (v3);
    \draw [-] (v2) -- (v0);
    \draw [-] (v2) -- (v4);
    \draw [-] (v2) -- (v1);

    \draw [-, very thick, color=red] (v3) .. controls (5,-1) and (5, 1) .. (v5);

    \draw [-, very thick, color=red] (v4) -- (v5);

    \draw [-, very thick, color=red] (v5) -- (v6);

	\node[] at ($ (c6) + (2.0,0.0) $) {\small 0-friend};
	\node[] at ($ (c5) + (2.0,0.0) $) {\small 1-friend};
	\node[] at ($ (c4) + (0.0,1.0) $) {\small 2-friend};
	\node[] at ($ (c3) + (2.0,0.0) $) {\small 2-friend};


\end{tikzpicture}
\caption{2-friend chain from node 6 \label{fig:tvenner}}
\end{figure}



\subsection{Algorithm}

To solve this problem, an obvois candiate is the
breadth-first search algorithm. In each iteration of this
search, the newly discovered nodes are one edge further 
away from the inital node. To accomplish the desired result, 
the algorithm just needs to keep track of iterations and
stop whenever the iteration counter exeeds $t$, and then
return all discovered nodes. The pseudocode can 
be seen in figure \ref{fig:tvennerpcode}.

\begin{figure}[ht]
\hrule
\vspace{0.2cm}
{\centering  \textit{tvenner}}
\vspace{0.2cm}
\hrule
\begin{algorithmic}
\Function{tvenner}{P, V, p, t} 
\State $G = $ \Call{incidentList}{P,V}
\State Mark all nodes as undiscovered
\State $Q = \{p\}$
\While{$Q \neq \emptyset$ \textbf{and} $t \geq 0$}
\State $L = \{\}$
\For{all nodes $v \in Q$}
	\State Mark $v$ as discovered
	\For{All $u$ pointed to by $v$}
		\State add $u$ to $L$
	\EndFor
\EndFor
\State $Q = L$
\State $t = t-1$
\EndWhile
\State \Return All marked notes
\EndFunction
\end{algorithmic}
\hrule
\caption{tvenner  \label{fig:tvennerpcode}}
\end{figure}

Instead of using a queue, this algorithm makes a temporary list 
for each iteration. The reson for this is that, unlike the 
regular Breadth-First algorithm, this needs finnish an entire 
iteration on a depth-level at a time and must distinguish 
each 'layer' of the graph.

\subsection{Complexity}

In the worst case of the algorithm, the runtime should be just as
the breadth first, since $t$ would, if sufficiently large, not 
affect how many nodes that needs to be searched. Thus the
worst-case time would be $\Theta \left( P + V \right)$








