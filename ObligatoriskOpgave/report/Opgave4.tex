\section{Assignment 4}

\subsection{Problem}

Now the relationships in the network has a weight indicating 
how string a relationship between two persons is. The problem 
to solve for this algorithm is finding a path between two nodes 
in the network that has the weakest relationships along the way.
If the algorithm is to find a chain of friends from $p_i$ to $p_j$,
it must choose a chain where the sum of the weight of the edges along 
the path is the least. The pseudocode can 
be seen in figure \ref{fig:shortchain}.



\begin{figure}[ht]
\centering
\begin{tikzpicture}[scale=0.7, auto,swap]
    \coordinate (c0) at (3, 0);
    \coordinate (c1) at (-3, -3);
    \coordinate (c2) at (0, 0);
    \coordinate (c3) at (3, -3);
    \coordinate (c4) at (-3, 3);
    \coordinate (c5) at (3, 3);
    \coordinate (c6) at (0, 6);

    \node[circle, draw, minimum size=\noderad] (v0) at (c0) {0};
    \node[circle, draw, minimum size=\noderad] (v1) at (c1) {1};
    \node[circle, draw, minimum size=\noderad] (v2) at (c2) {2};
    \node[circle, draw, minimum size=\noderad] (v3) at (c3) {3};
    \node[circle, draw, minimum size=\noderad] (v4) at (c4) {4};
    \node[circle, draw, minimum size=\noderad] (v5) at (c5) {5};
    \node[circle, draw, minimum size=\noderad] (v6) at (c6) {6};

    \draw [-] (v1) -- (v2);
    \draw [-, very thick, color=red] (v1) -- (v4);

    \draw [-] (v2) -- (v3);
    \draw [-] (v2) -- (v0);
    \draw [-] (v2) -- (v4);
    \draw [-] (v2) -- (v1);

    \draw [-] (v3) .. controls (5,-1) and (5, 1) .. (v5);

    \draw [-, very thick, color=red] (v4) -- (v5);

    \draw [-, very thick, color=red] (v5) -- (v6);

	\node[] at ($ (c0) + (-1.5,0.5) $) {\small 5};
	\node[] at ($ (c1) + (1.3,1.9) $) {\small 2};
	\node[] at ($ (c5) + (-1.25,2.0) $) {\small 88};
	\node[] at ($ (c3) + (2.3,2.8) $) {\small 77};
	\node[] at ($ (c2) + (-0.8,1.7) $) {\small 40};
	\node[] at ($ (c1) + (-0.5,2.5) $) {\small 5};
	\node[] at ($ (c2) + (1.8,-1.2) $) {\small 8};
	\node[] at ($ (c4) + (3.0,0.5) $) {\small 10};



\end{tikzpicture}
\caption{Weakest friends chain between node 6 and 1 \label{fig:shortchain}}
\end{figure}

\subsection{Algorithm}

To solve the problem, one can almost just use Dijkstra
shortest path algorithm as it is. If the algorithm is run
and, afterwards, the path from $p_1$ to $p_2$ can be found
by following the nodes pointed to by starting from $o_2$ and 
take the reverse order. The pseudocode can 
be seen in figure \ref{fig:shortpathcode}.


\begin{figure}[ht!]
\hrule
\vspace{0.2cm}
{\centering  \textit{svagestevenskab}}
\vspace{0.2cm}
\hrule
\begin{algorithmic}
\Function{svagestevenskab}{P, V, p1, p2} 
\State $G = $ \Call{incidentList}{P,V}
\State \Call{Dijkstra}{G, p1}
\State $N = \{p2\}$
\State $n = p2$
\While{$n.\pi \neq null$}
\State $N.$\Call{add}{$n.\pi$}
\State $n = n.\pi$
\EndWhile
\State \Return $N.$\Call{reverse}{}
\EndFunction
\end{algorithmic}
\hrule
\caption{svagestevenskab  \label{fig:shortpathcode}}
\end{figure}

\subsection{Complexity}

Folowing the path adds at most $P$ operations, and so does revering.
Thus besides from the Dijkstra's algorithm itself, only $2P$ operations
are added. Thus, he runtime is: $T(P, V, p1, p2) = \Theta \left( V + P \right)$




