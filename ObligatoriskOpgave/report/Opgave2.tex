\section{Problem 2}


\subsection{Problem}

The input is given in the same way as in the prior assignment.
The algorithm for this assignment is to find out if a set 
of persons $X = {x_0, x_1, \dots, x_t}, \; X \subset P$  in the network is 
in a close friendship.
Any such subset is is said to be a close friendship if any person in $X$
has a friendship relation to \textit{all other} persons in $X$. An example
of a close friendship can be seen in figure \ref{fig:closefship}.

\begin{figure}[ht]
\centering
\begin{tikzpicture}[scale=0.7, auto,swap]
    \coordinate (c0) at (3, 0);
    \coordinate (c1) at (-3, -3);
    \coordinate (c2) at (0, 0);
    \coordinate (c3) at (3, -3);
    \coordinate (c4) at (-3, 3);
    \coordinate (c5) at (3, 3);
    \coordinate (c6) at (0, 6);

    \node[circle, draw, minimum size=\noderad] (v0) at (c0) {0};
    \node[circle, draw, minimum size=\noderad] (v1) at (c1) {1};
    \node[circle, draw, minimum size=\noderad] (v2) at (c2) {2};
    \node[circle, draw, minimum size=\noderad] (v3) at (c3) {3};
    \node[circle, draw, minimum size=\noderad] (v4) at (c4) {4};
    \node[circle, draw, minimum size=\noderad] (v5) at (c5) {5};
    \node[circle, draw, minimum size=\noderad] (v6) at (c6) {6};

    \draw [-] (v1) -- (v2);
    \draw [-, very thick, color=red] (v1) -- (v4);

    \draw [-] (v2) -- (v3);
    \draw [-] (v2) -- (v0);
    \draw [-, very thick, color=red] (v2) -- (v4);
    \draw [-, very thick, color=red] (v2) -- (v1);

    \draw [-] (v3) .. controls (5,-1) and (5, 1) .. (v5);

    \draw [-] (v4) -- (v5);

    \draw [-] (v5) -- (v6);



\end{tikzpicture}
\caption{Close frindship between node 1,2 and 4 \label{fig:closefship}}
\end{figure}


\subsection{Algorithm}

In the source code, the function that determines if a set of
people are in a close friendship is named \textit{taet\_venskab}.
The pseudocode can be seen in figure \ref{fig:cla}.\\

\begin{figure}[ht]
\hrule
\vspace{0.2cm}
{\centering  \textit{taet\_venskab}}
\vspace{0.2cm}
\hrule
\begin{algorithmic}
\Function{taet\_venskab}{P, V, X} 
\State $G = $ \Call{incidentList}{P,V}
\State $t = X.length$
\For{$i = 0$ to $t-1$}
	\For{$j = i+1$ to $t-1$}
		\If{\textbf{not} $X[j] \in G[X[i]]$}
			\State \Return "nej"
		\EndIf
	\EndFor
\EndFor
\State \Return "ja"
\EndFunction
\end{algorithmic}
\hrule
\caption{Close friendship algorithm \label{fig:cla}}
\end{figure}


The Algorithm uses an incident list to represent the graph and albiet 
occuring in the algorithm, the creation of theincident list is not 
accounted for in the complexity analasys.\\

To compute Weather the people are in a close friendship, the algorithm
check for each person if that person has an edge in the graph to
every other person in the graph. If, at any point the algorithm encounters
a person that lacks a edge to another person, the algorithm returns with "nej",
otherwise, the algorithm returns "yes".

\subsection{Complexity}

As mentioned before, the complexity analasys does not account for the 
time it takes to build the incident list of the graph. However, this
is assumed to $\Theta \left( V + P \right)$.\\

In the core algorithm, there are two loops. The code of the 
inner-for loop can be seen much like how \textbf{BubbleSort}
runs. The amount of times the inner-loops code is executed is

\begin{align*}
k-1 + k-2 + \dots + k-k &= \sum^{k-1}_{i=0} i \\
 & = \frac{k^2-k}{2} \in \Theta \left( k^2 \right)
\end{align*} 

Inside the loop, the \textbf{if} statement checks if
the incident list of index $i$ contains an edge to $person$
$x_j$. In worst case, this would require traversing through 
alledges of $G[X[i]]$.
In a worst case scenario, would be $|P|-1$ as $x_i$ could
have at most once relation to all other persons in the 
network.

\begin{align*}
T(P, V, N, k) &= c_1 \cdot \frac{k^2-k}{2} \cdot c_2 \cdot \left( |P|-1 \right)\\
&= \Theta \left( k^2 \cdot |P| \right)
\end{align*}

Again, this only accounts for the core algorithm, and not 
the overhead of constructing the graph representation.\\

If a matrix were used to represent the graph, the runtime could have
been reduced to $\Theta \left( k^2 \right)$ due to random access
lookup for edges. However, a matrix representation of the graph 
prooved too costly regarding memory usage.







